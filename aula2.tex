\documentclass{beamer}
\usepackage[utf8]{inputenc}

\title{Computação em Nuvem}
\subtitle{Aula 2}

\usetheme{lucid}

\begin{document}
\frame{
 \titlepage
}

\frame{
    \frametitle{Roteiro de Aula}
    \tableofcontents
}
\section{Modelos de Serviço}
\frame {
 \frametitle{Modelos de serviço em computação em nuvem}
 O modelo de serviço define o nível de controle que o cliente tem sobre os recursos alocados no provedor.
\begin{itemize}
    \item \textbf{(IaaS)}: Infraestrutura como Serviço;
    \item \textbf{PaaS}: Plataforma como Serviço;
    \item \textbf{SaaS}: Software como Serviço.
\end{itemize}
}
\frame {
 \frametitle{Modelo IaaS}
 \framesubtitle{Breve apresentação}
 
 \begin{block}{Características}
    \begin{itemize}
     \item O cliente pode alocar capacidade computacional na forma de recursos computacionais virtualizados. 
     \item Assim, o cliente pode alocar, no provedor, uma infraestrutura, com a capacidade desejada de processamento e armazenamento de dados. 
     \item A alocação desses recursos possibilita que o cliente tenha controle da escolha do sistema operacional que será utilizado nos servidores alocados.
    \end{itemize}
 \end{block}
}

\frame {
 \frametitle{Modelo PaaS}
\begin{block}{Características}
    \begin{enumerate}
        \item O cliente não tem controle sobre a infraestrutura. 
        \item Cliente recebe do provedor um ambiente já configurado, pronto para o desenvolvimento de aplicações. 
        \item O cliente escolhe apenas a plataforma de desenvolvimento, tais como, linguagem de programação e banco de dados.
    \end{enumerate}
    \end{block}
} 

\frame {
 \frametitle{Modelo SaaS}
\begin{block}{Características}
    \begin{enumerate}
        \item O cliente acessa aplicações;
        \item Cliente não tem nenhum controle sobre os recursos computacionais disponíveis no provedor.
    \end{enumerate}
    \end{block}
} 
\section{Abstração}
\frame{
\frametitle{Pontos importantes}
\framesubtitle{O que avaliar?}

\begin{enumerate}
    \item Uma das principais características da computação em nuvem é a abstração da complexidade dos recursos computacionais alocados. 
    \item O nível de abstração da complexidade está relacionado ao nível de controle do cliente. 
\end{enumerate}
\textit{Ao escolher um modelo de serviço com menor controle do cliente sobre a infraestrutura subjacente, maior é o nível de abstração da complexidade. }
}
 
\frame{
\frametitle{Fundamentos de computação em nuvem}
\framesubtitle{Vamos jogar!}
\begin{center}
    Acesse Kahoot: www.kahoot.it\\
    Digite o código: na tela.
\end{center}
}

\frame{
\frametitle{Atividade de aula}
\framesubtitle{Formulário ativo até as 22:30h}
\begin{center}
   % \includegraphics[width=0.4\linewidth]{fig/aula1/QRCode para Computação em nuvem.png}
\end{center}
}

\end{document}