\documentclass{beamer}
\usepackage[utf8]{inputenc}
\usepackage{hyperref}

\title{Computação em Nuvem - Aula 4}
\subtitle{Fundamentação teórica\\
Prof. Me. Juliana Costa-Silva}

\usetheme{lucid}

\begin{document}
\frame{
 \titlepage
}

\frame{
    \frametitle{Roteiro de Aula}
    \tableofcontents
}
\section{Atvidade}
\frame {
 \frametitle{Atividade Parcial}
 \framesubtitle{Leitura e reescrita}
 Escolha 1 dos artigos para a leitura:
 \begin{block}{Artigo: }
 \begin{enumerate}
 % No artigo indicado a seguir, é feito um estudo que compara o custo da aquisição de servidores físicos com o custo do uso de máquinas virtuais em um provedor de nuvem pública. O cenário hipotético apresentado ilustra diversas variáveis envolvidas na análise dos custos, por exemplo, a taxa de utilização dos servidores.
     \item \textbf{\href{https://github.com/costasilvati/CompuNuvem/blob/main/Materiais_de_apoio/computacao_nuvem-with-cover-page.pdf}{Computação em Nuvem: Conceitos, Tecnologias, Aplicações e Desafios}};
     \item \textbf{\href{https://github.com/costasilvati/CompuNuvem/blob/main/Materiais_de_apoio/THOME_ERRC_2013.pdf}{Computação em Nuvem: Análise Comparativa de Ferramentas Open Source para IaaS}}
    \item \textbf{\href{https://github.com/costasilvati/CompuNuvem/blob/main/Materiais_de_apoio/EstudocomparativosobreousodoVMwareeXenServernavirtualizaodeServidores.pdf}{Estudo comparativo sobre o uso do VMware e Xen Server na virtualização de Servidores}}
 \end{enumerate}
 \end{block}
}
%----------------------------------------
\section{Avaliação}
\frame {
 \frametitle{Prepare-se}
 \framesubtitle{Vídeo ou animação}
 Você deve preparar um vídeo de no máximo 5 minutos, onde você apresenta:
    \begin{enumerate}
        \item O artigo escolhido;
        \item Sobre o que o artigo fala?
        \item Como foram realizadas as comparações apresentadas?
        \item Quais as principais contribuições do artigo? (mínimo 3)
        \item Qual a conclusão do artigo? Depois de todas as análises o que os autores dizem sobre o que foi analisado?
    \end{enumerate}
    Envie como resposta da atividade o link para o seu vídeo (YouTube ou outra plataforma);
} 
%----------------------------------------
\begin{frame}{Referências}%[allowframebreaks]
 \tiny
 \begin{center}
 	\bibliographystyle{apalike}
	 \bibliography{ref}
 \end{center}
 \end{frame}

\end{document}