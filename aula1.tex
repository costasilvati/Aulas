\documentclass{beamer}
\usepackage[utf8]{inputenc}

\title{Computação em Nuvem}
\subtitle{Aula 1}

\usetheme{lucid}

\begin{document}
\frame{
 \titlepage
}

\frame{
    \frametitle{Roteiro de Aula}
    \tableofcontents
}
\section{Apresentação}
\frame {
 \frametitle{Prof. Me. Juliana Costa-Silva}
\begin{itemize}
    \item \textbf{Graduação:} Análise e Desenvolvimento de Sistemas - 2011 a 2014 - UTFPR-CP;
    \begin{itemize}
        \item Estágio: ECD Multcabos (Tec.) / Hayamax (Dev)/ IAPAR - Lab. Biotecnologia Vegetal (Bioinformática).
    \end{itemize}
    \item \textbf{Mestrado:} Bioinformática - 2017 - UTFPR-CP;
    \item \textbf{Docente:} 
        \begin{itemize}
            \item Universidade Positivo (2017 - atual);
            \item Unicesumar (2017 - 2020);
            \item IFPR (2018 - 2019);
            \item Univesidade Brasil EAD (2018);
            \item SENAI (2020).
        \end{itemize}
    \item \textbf{Doutoranda:} Informática - 2020 até 2024 - UFPR (Curitiba).
\end{itemize}
}
\frame {
 \frametitle{Quem é o aluno?}
 \framesubtitle{Breve apresentação}
 Responda:
 \begin{itemize}
     \item Qual o seu nome?  
     \item Qual a sua idade?
     \item Qual a sua profissão?
     \item Matéria que mais gostou da graduação?
 \end{itemize}
}

\section{Plano de Ensino}

\frame {
 \frametitle{O que veremos?}
    \textbf{Bimestre 1}
    \begin{enumerate}
        \item Conceitos básicos em computação em nuvem;
        \item Modelos de serviço em computação em nuvem;
        \item Modelos de implantação em computação em nuvem;
        \item Tecnologias de suporte à nuvem;
        \item Provedores de computação em nuvem;
        \item Migração de aplicações para a nuvem;
    \end{enumerate}
} 
\frame{
\frametitle{O que veremos?}
\framesubtitle{Parte 2}
\textbf{Bimestre 2}
\begin{enumerate}
    \item Serviços de processamento de dados;   \item Serviços de armazenamento e análise de dados;
    \item Soluções em nuvem;
    \item Modelos de arquitetura em nuvem;
    \item Qualidade de serviço em nuvem;
    \item Segurança e privacidade em nuvem 
\end{enumerate}
}

\frame{
\frametitle{Fundamentos de computação em nuvem}
\framesubtitle{Vamos jogar!}
\begin{center}
    \includegraphics[width=0.4\linewidth]{fig/aula1/frame.png}
\end{center}
}

\frame{
\frametitle{Atividade de aula}
\framesubtitle{Formulário ativo até as 23:59h}
\begin{center}
    \includegraphics[width=0.4\linewidth]{fig/aula1/QRCode para Computação em nuvem.png}
\end{center}
}

\end{document}